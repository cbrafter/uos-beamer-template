\documentclass{beamer}
\usetheme{craiguos}
\usepackage[utf8]{inputenc}
\usepackage[main=english]{babel}        

%% These macros specify information about the presentation
\title{Main Title}
\subtitle{Subtitle}
\date{\today}
\author{Craig B. Rafter}
\institute{Transportation Research Group}
\titlegraphic{\hfill\includegraphics[height=1.0cm]{uos-logo-white.png}}

%% These additional packages are used within the document:
\usepackage{ragged2e}  % `\justifying` text
\usepackage{booktabs}  % Tables
\usepackage{tabularx}
\usepackage{tikz}      % Diagrams
\usetikzlibrary{calc, shapes, backgrounds}
\usepackage{amsmath, amssymb}
\usepackage{url}       % `\url`s
\usepackage{listings}  % Code listings
\usepackage{color}

\frenchspacing
\begin{document}
	\setbeamercolor{background canvas}{bg=uosblue}
	\begin{frame}[plain]
		\titlepage
	\end{frame}
	\setbeamercolor{background canvas}{bg=}
	
	  \AtBeginSection[]{% Print an outline at the beginning of sections
	  	\begin{frame}<beamer>
	  		\frametitle{Outline for Section \thesection}
	  		\tableofcontents[currentsection]
	  	\end{frame}
	  }	
	
	\section{First}
	\begin{frame} 
		\frametitle{There Is No Largest Prime Number} 
		\framesubtitle{The proof uses \textit{reductio ad absurdum}.} 
		\begin{theorem}
			There is no largest prime number. \end{theorem} 
		\begin{enumerate} 
			\item<1-| alert@1> Suppose $p$ were the largest prime number. 
			\item<2-> Let $q$ be the product of the first $p$ numbers. 
			\item<3-> Then $q+1$ is not divisible by any of them. 
			\item<1-> But $q + 1$ is greater than $1$, thus divisible by some prime
			number not in the first $p$ numbers.
		\end{enumerate}
	\end{frame}
	
	\section{Second}
	\begin{frame}{A longer title}
		\begin{itemize}
			\item one
			\item two
		\end{itemize}
	\end{frame}
	
\end{document}